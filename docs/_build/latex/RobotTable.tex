% Generated by Sphinx.
\def\sphinxdocclass{report}
\documentclass[letterpaper,10pt,english]{sphinxmanual}
\usepackage[utf8]{inputenc}
\DeclareUnicodeCharacter{00A0}{\nobreakspace}
\usepackage{cmap}
\usepackage[T1]{fontenc}
\usepackage{babel}
\usepackage{times}
\usepackage[Bjarne]{fncychap}
\usepackage{longtable}
\usepackage{sphinx}
\usepackage{multirow}


\title{Robot Table Documentation}
\date{August 03, 2013}
\release{1.0}
\author{Arnaud}
\newcommand{\sphinxlogo}{}
\renewcommand{\releasename}{Release}
\makeindex

\makeatletter
\def\PYG@reset{\let\PYG@it=\relax \let\PYG@bf=\relax%
    \let\PYG@ul=\relax \let\PYG@tc=\relax%
    \let\PYG@bc=\relax \let\PYG@ff=\relax}
\def\PYG@tok#1{\csname PYG@tok@#1\endcsname}
\def\PYG@toks#1+{\ifx\relax#1\empty\else%
    \PYG@tok{#1}\expandafter\PYG@toks\fi}
\def\PYG@do#1{\PYG@bc{\PYG@tc{\PYG@ul{%
    \PYG@it{\PYG@bf{\PYG@ff{#1}}}}}}}
\def\PYG#1#2{\PYG@reset\PYG@toks#1+\relax+\PYG@do{#2}}

\expandafter\def\csname PYG@tok@gu\endcsname{\let\PYG@bf=\textbf\def\PYG@tc##1{\textcolor[rgb]{0.50,0.00,0.50}{##1}}}
\expandafter\def\csname PYG@tok@gt\endcsname{\let\PYG@bf=\textbf\def\PYG@tc##1{\textcolor[rgb]{0.64,0.00,0.00}{##1}}}
\expandafter\def\csname PYG@tok@gs\endcsname{\let\PYG@bf=\textbf\def\PYG@tc##1{\textcolor[rgb]{0.00,0.00,0.00}{##1}}}
\expandafter\def\csname PYG@tok@gr\endcsname{\def\PYG@tc##1{\textcolor[rgb]{0.94,0.16,0.16}{##1}}}
\expandafter\def\csname PYG@tok@cm\endcsname{\let\PYG@it=\textit\def\PYG@tc##1{\textcolor[rgb]{0.56,0.35,0.01}{##1}}}
\expandafter\def\csname PYG@tok@gp\endcsname{\def\PYG@tc##1{\textcolor[rgb]{0.45,0.33,0.20}{##1}}}
\expandafter\def\csname PYG@tok@m\endcsname{\def\PYG@tc##1{\textcolor[rgb]{0.60,0.00,0.00}{##1}}}
\expandafter\def\csname PYG@tok@na\endcsname{\def\PYG@tc##1{\textcolor[rgb]{0.77,0.63,0.00}{##1}}}
\expandafter\def\csname PYG@tok@go\endcsname{\def\PYG@tc##1{\textcolor[rgb]{0.53,0.53,0.53}{##1}}}
\expandafter\def\csname PYG@tok@ge\endcsname{\let\PYG@it=\textit\def\PYG@tc##1{\textcolor[rgb]{0.00,0.00,0.00}{##1}}}
\expandafter\def\csname PYG@tok@gd\endcsname{\def\PYG@tc##1{\textcolor[rgb]{0.64,0.00,0.00}{##1}}}
\expandafter\def\csname PYG@tok@il\endcsname{\def\PYG@tc##1{\textcolor[rgb]{0.60,0.00,0.00}{##1}}}
\expandafter\def\csname PYG@tok@cs\endcsname{\let\PYG@it=\textit\def\PYG@tc##1{\textcolor[rgb]{0.56,0.35,0.01}{##1}}}
\expandafter\def\csname PYG@tok@cp\endcsname{\def\PYG@tc##1{\textcolor[rgb]{0.56,0.35,0.01}{##1}}}
\expandafter\def\csname PYG@tok@gi\endcsname{\def\PYG@tc##1{\textcolor[rgb]{0.00,0.63,0.00}{##1}}}
\expandafter\def\csname PYG@tok@gh\endcsname{\let\PYG@bf=\textbf\def\PYG@tc##1{\textcolor[rgb]{0.00,0.00,0.50}{##1}}}
\expandafter\def\csname PYG@tok@ni\endcsname{\def\PYG@tc##1{\textcolor[rgb]{0.81,0.36,0.00}{##1}}}
\expandafter\def\csname PYG@tok@ld\endcsname{\def\PYG@tc##1{\textcolor[rgb]{0.00,0.00,0.00}{##1}}}
\expandafter\def\csname PYG@tok@nl\endcsname{\def\PYG@tc##1{\textcolor[rgb]{0.96,0.47,0.00}{##1}}}
\expandafter\def\csname PYG@tok@nn\endcsname{\def\PYG@tc##1{\textcolor[rgb]{0.00,0.00,0.00}{##1}}}
\expandafter\def\csname PYG@tok@no\endcsname{\def\PYG@tc##1{\textcolor[rgb]{0.00,0.00,0.00}{##1}}}
\expandafter\def\csname PYG@tok@s2\endcsname{\def\PYG@tc##1{\textcolor[rgb]{0.31,0.60,0.02}{##1}}}
\expandafter\def\csname PYG@tok@nb\endcsname{\def\PYG@tc##1{\textcolor[rgb]{0.00,0.27,0.38}{##1}}}
\expandafter\def\csname PYG@tok@nc\endcsname{\def\PYG@tc##1{\textcolor[rgb]{0.00,0.00,0.00}{##1}}}
\expandafter\def\csname PYG@tok@nd\endcsname{\def\PYG@tc##1{\textcolor[rgb]{0.53,0.53,0.53}{##1}}}
\expandafter\def\csname PYG@tok@ne\endcsname{\let\PYG@bf=\textbf\def\PYG@tc##1{\textcolor[rgb]{0.80,0.00,0.00}{##1}}}
\expandafter\def\csname PYG@tok@nf\endcsname{\def\PYG@tc##1{\textcolor[rgb]{0.00,0.00,0.00}{##1}}}
\expandafter\def\csname PYG@tok@nx\endcsname{\def\PYG@tc##1{\textcolor[rgb]{0.00,0.00,0.00}{##1}}}
\expandafter\def\csname PYG@tok@si\endcsname{\def\PYG@tc##1{\textcolor[rgb]{0.31,0.60,0.02}{##1}}}
\expandafter\def\csname PYG@tok@sh\endcsname{\def\PYG@tc##1{\textcolor[rgb]{0.31,0.60,0.02}{##1}}}
\expandafter\def\csname PYG@tok@vi\endcsname{\def\PYG@tc##1{\textcolor[rgb]{0.00,0.00,0.00}{##1}}}
\expandafter\def\csname PYG@tok@py\endcsname{\def\PYG@tc##1{\textcolor[rgb]{0.00,0.00,0.00}{##1}}}
\expandafter\def\csname PYG@tok@nt\endcsname{\let\PYG@bf=\textbf\def\PYG@tc##1{\textcolor[rgb]{0.00,0.27,0.38}{##1}}}
\expandafter\def\csname PYG@tok@nv\endcsname{\def\PYG@tc##1{\textcolor[rgb]{0.00,0.00,0.00}{##1}}}
\expandafter\def\csname PYG@tok@s1\endcsname{\def\PYG@tc##1{\textcolor[rgb]{0.31,0.60,0.02}{##1}}}
\expandafter\def\csname PYG@tok@vg\endcsname{\def\PYG@tc##1{\textcolor[rgb]{0.00,0.00,0.00}{##1}}}
\expandafter\def\csname PYG@tok@kd\endcsname{\let\PYG@bf=\textbf\def\PYG@tc##1{\textcolor[rgb]{0.00,0.27,0.38}{##1}}}
\expandafter\def\csname PYG@tok@vc\endcsname{\def\PYG@tc##1{\textcolor[rgb]{0.00,0.00,0.00}{##1}}}
\expandafter\def\csname PYG@tok@ow\endcsname{\let\PYG@bf=\textbf\def\PYG@tc##1{\textcolor[rgb]{0.00,0.27,0.38}{##1}}}
\expandafter\def\csname PYG@tok@mf\endcsname{\def\PYG@tc##1{\textcolor[rgb]{0.60,0.00,0.00}{##1}}}
\expandafter\def\csname PYG@tok@bp\endcsname{\def\PYG@tc##1{\textcolor[rgb]{0.20,0.40,0.64}{##1}}}
\expandafter\def\csname PYG@tok@x\endcsname{\def\PYG@tc##1{\textcolor[rgb]{0.00,0.00,0.00}{##1}}}
\expandafter\def\csname PYG@tok@c1\endcsname{\let\PYG@it=\textit\def\PYG@tc##1{\textcolor[rgb]{0.56,0.35,0.01}{##1}}}
\expandafter\def\csname PYG@tok@o\endcsname{\def\PYG@tc##1{\textcolor[rgb]{0.35,0.16,0.00}{##1}}}
\expandafter\def\csname PYG@tok@kc\endcsname{\let\PYG@bf=\textbf\def\PYG@tc##1{\textcolor[rgb]{0.00,0.27,0.38}{##1}}}
\expandafter\def\csname PYG@tok@c\endcsname{\let\PYG@it=\textit\def\PYG@tc##1{\textcolor[rgb]{0.56,0.35,0.01}{##1}}}
\expandafter\def\csname PYG@tok@kr\endcsname{\let\PYG@bf=\textbf\def\PYG@tc##1{\textcolor[rgb]{0.00,0.27,0.38}{##1}}}
\expandafter\def\csname PYG@tok@g\endcsname{\def\PYG@tc##1{\textcolor[rgb]{0.00,0.00,0.00}{##1}}}
\expandafter\def\csname PYG@tok@err\endcsname{\def\PYG@tc##1{\textcolor[rgb]{0.64,0.00,0.00}{##1}}\def\PYG@bc##1{\setlength{\fboxsep}{0pt}\fcolorbox[rgb]{0.94,0.16,0.16}{1,1,1}{\strut ##1}}}
\expandafter\def\csname PYG@tok@mh\endcsname{\def\PYG@tc##1{\textcolor[rgb]{0.60,0.00,0.00}{##1}}}
\expandafter\def\csname PYG@tok@ss\endcsname{\def\PYG@tc##1{\textcolor[rgb]{0.31,0.60,0.02}{##1}}}
\expandafter\def\csname PYG@tok@sr\endcsname{\def\PYG@tc##1{\textcolor[rgb]{0.31,0.60,0.02}{##1}}}
\expandafter\def\csname PYG@tok@mo\endcsname{\def\PYG@tc##1{\textcolor[rgb]{0.60,0.00,0.00}{##1}}}
\expandafter\def\csname PYG@tok@kn\endcsname{\let\PYG@bf=\textbf\def\PYG@tc##1{\textcolor[rgb]{0.00,0.27,0.38}{##1}}}
\expandafter\def\csname PYG@tok@mi\endcsname{\def\PYG@tc##1{\textcolor[rgb]{0.60,0.00,0.00}{##1}}}
\expandafter\def\csname PYG@tok@l\endcsname{\def\PYG@tc##1{\textcolor[rgb]{0.00,0.00,0.00}{##1}}}
\expandafter\def\csname PYG@tok@sx\endcsname{\def\PYG@tc##1{\textcolor[rgb]{0.31,0.60,0.02}{##1}}}
\expandafter\def\csname PYG@tok@n\endcsname{\def\PYG@tc##1{\textcolor[rgb]{0.00,0.00,0.00}{##1}}}
\expandafter\def\csname PYG@tok@p\endcsname{\let\PYG@bf=\textbf\def\PYG@tc##1{\textcolor[rgb]{0.00,0.00,0.00}{##1}}}
\expandafter\def\csname PYG@tok@s\endcsname{\def\PYG@tc##1{\textcolor[rgb]{0.31,0.60,0.02}{##1}}}
\expandafter\def\csname PYG@tok@kp\endcsname{\let\PYG@bf=\textbf\def\PYG@tc##1{\textcolor[rgb]{0.00,0.27,0.38}{##1}}}
\expandafter\def\csname PYG@tok@w\endcsname{\let\PYG@ul=\underline\def\PYG@tc##1{\textcolor[rgb]{0.97,0.97,0.97}{##1}}}
\expandafter\def\csname PYG@tok@kt\endcsname{\let\PYG@bf=\textbf\def\PYG@tc##1{\textcolor[rgb]{0.00,0.27,0.38}{##1}}}
\expandafter\def\csname PYG@tok@sc\endcsname{\def\PYG@tc##1{\textcolor[rgb]{0.31,0.60,0.02}{##1}}}
\expandafter\def\csname PYG@tok@sb\endcsname{\def\PYG@tc##1{\textcolor[rgb]{0.31,0.60,0.02}{##1}}}
\expandafter\def\csname PYG@tok@k\endcsname{\let\PYG@bf=\textbf\def\PYG@tc##1{\textcolor[rgb]{0.00,0.27,0.38}{##1}}}
\expandafter\def\csname PYG@tok@se\endcsname{\def\PYG@tc##1{\textcolor[rgb]{0.31,0.60,0.02}{##1}}}
\expandafter\def\csname PYG@tok@sd\endcsname{\let\PYG@it=\textit\def\PYG@tc##1{\textcolor[rgb]{0.56,0.35,0.01}{##1}}}

\def\PYGZbs{\char`\\}
\def\PYGZus{\char`\_}
\def\PYGZob{\char`\{}
\def\PYGZcb{\char`\}}
\def\PYGZca{\char`\^}
\def\PYGZam{\char`\&}
\def\PYGZlt{\char`\<}
\def\PYGZgt{\char`\>}
\def\PYGZsh{\char`\#}
\def\PYGZpc{\char`\%}
\def\PYGZdl{\char`\$}
\def\PYGZhy{\char`\-}
\def\PYGZsq{\char`\'}
\def\PYGZdq{\char`\"}
\def\PYGZti{\char`\~}
% for compatibility with earlier versions
\def\PYGZat{@}
\def\PYGZlb{[}
\def\PYGZrb{]}
\makeatother

\begin{document}

\maketitle
\tableofcontents
\phantomsection\label{index::doc}


In 2002, Lincoln University, with the collaboration of Tufts University, started to develop a Robotable environment to enhance learning about robotics and engineering problem solving.

A Robotable is a tabletop that acts as a rear-projection screen. A picture is projected on it using a mirror (45 degrees inclined) and a projector. The system is completed by an optical tracking system used to detect interaction on the tabletop.

In 2012, Leshi Chen, a master student at Lincoln University, worked to improve the Robotable environment. His goal was to facilitate the use of Robotables by providing a set of toolkits (in C\#) for setting-up and creating easily new games.

My project was to port Leshi's toolkits to a more portable computer: a Raspberry Pi. With a Raspberry Pi, setting-up Robotables environment will be easier thanks to its low price and small size. In fact, rather than using a standard computer (with a monitor, keyboard...) we will be able to use a computer with the size of a credit card.

The easiest way to port the toolkits was to rewrite them in another programming language: Python.

This documentation explains how to use the software on a Raspberry Pi.


\chapter{User Guide}
\label{index:user-guide}\label{index:welcome-to-robotable-s-documentation}
This part of the documentation explains how the software works
and how to run the game example on your own robot table.


\section{Introduction}
\label{user/intro:introduction}\label{user/intro::doc}

\subsection{What's a Robot Table?}
\label{user/intro:what-s-a-robot-table}
Add a picture here.


\subsection{Requirements}
\label{user/intro:requirements}\label{user/intro:id1}

\subsubsection{Hardware}
\label{user/intro:hardware}\begin{itemize}
\item {} 
Robot Table

\item {} 
Projector

\item {} 
\href{http://www.raspberrypi.org/}{Rasberry Pi}

\item {} 
\href{http://www.amazon.com/Wii-Remote-Controller-Nintendo/dp/B000IMWK2G}{Wiimote}

\item {} 
Dongle USB Bluetooth

\end{itemize}


\subsubsection{Software}
\label{user/intro:software}\begin{itemize}
\item {} 
\href{https://github.com/abstrakraft/cwiid}{cwiid}

\item {} 
\href{http://docs.python.org/2/library/tkinter.html}{Tkinter}

\item {} 
\href{http://flask.pocoo.org/}{Flask}

\item {} 
\href{http://docs.python-requests.org/en/latest/\#}{Requests}

\item {} 
\href{http://effbot.org/imagingbook/imagetk.htm}{ImageTk}

\end{itemize}

More information on how to install these software can be found {\hyperref[user/install:installrequirements]{\emph{here}}}.


\section{Installation}
\label{user/install:installation}\label{user/install::doc}

\subsection{Get the Code}
\label{user/install:get-the-code}
The code is available on \href{https://github.com/arnaudchenyensu/RoboTableProject}{Github}. You can easily clone the repository:

\begin{Verbatim}[commandchars=\\\{\}]
\$ git clone https://github.com/arnaudchenyensu/RoboTableProject.git
\end{Verbatim}


\subsection{Installing the requirements}
\label{user/install:installing-the-requirements}\label{user/install:installrequirements}
The library \href{http://goo.gl/RYfUPk}{cwiid}, \href{http://goo.gl/XlP9g9}{ImageTk} and for the dongle USB bluetooth can be installed using apt-get:

\begin{Verbatim}[commandchars=\\\{\}]
\$ sudo apt-get install python-cwiid python-imaging python-imaging-tk bluetooth bluez-utils blueman
\end{Verbatim}

The other softwares can usually be installed using \href{https://pypi.python.org/pypi/pip}{pip} or \href{https://pypi.python.org/pypi/setuptools}{easy\_install}.

You will find more information on each software's website.


\section{Quickstart}
\label{user/quickstart::doc}\label{user/quickstart:quickstart}
When every {\hyperref[user/intro:requirements]{\emph{requirements}}} are met, you'll need to launch the X server:

\begin{Verbatim}[commandchars=\\\{\}]
\$ startx
\$ export DISPLAY=:0.0
\end{Verbatim}

Then, you can launch the server using the script server.py:

\begin{Verbatim}[commandchars=\\\{\}]
\$ python server.py
\end{Verbatim}

The web server is available at the Raspberry Pi's IP address on the port 5000. If the ip address of your Raspberry is 10.4.9.4, you can access the server at \href{http://10.4.9.4:5000}{http://10.4.9.4:5000}. You can then launch your game using the button `Start game'.


\chapter{Developer Guide}
\label{index:developer-guide}
This part of the documentation explains how to write a new game
using the existing scripts.


\section{How to create a new game?}
\label{developer/create_new_game:how-to-create-a-new-game}\label{developer/create_new_game::doc}
If you want to create a new game, you will have to create a new class that inherits from the Game class. Then you need to override the method start. When you override this method, you'll need first to call these methods:

\begin{Verbatim}[commandchars=\\\{\}]
\PYG{n+nb+bp}{self}\PYG{o}{.}\PYG{n}{game\PYGZus{}management}\PYG{o}{.}\PYG{n}{start}\PYG{p}{(}\PYG{n+nb+bp}{self}\PYG{o}{.}\PYG{n}{addr\PYGZus{}main\PYGZus{}server}\PYG{p}{,} \PYG{n+nb+bp}{self}\PYG{o}{.}\PYG{n}{is\PYGZus{}main\PYGZus{}server}\PYG{p}{(}\PYG{p}{)}\PYG{p}{,} \PYG{n+nb+bp}{self}\PYG{o}{.}\PYG{n}{addr}\PYG{p}{)}
\PYG{n+nb+bp}{self}\PYG{o}{.}\PYG{n}{x\PYGZus{}factors} \PYG{o}{=} \PYG{n+nb+bp}{self}\PYG{o}{.}\PYG{n}{game\PYGZus{}management}\PYG{o}{.}\PYG{n}{x\PYGZus{}factors}
\PYG{n+nb+bp}{self}\PYG{o}{.}\PYG{n}{y\PYGZus{}factors} \PYG{o}{=} \PYG{n+nb+bp}{self}\PYG{o}{.}\PYG{n}{game\PYGZus{}management}\PYG{o}{.}\PYG{n}{y\PYGZus{}factors}
\PYG{n+nb+bp}{self}\PYG{o}{.}\PYG{n}{servers} \PYG{o}{=} \PYG{n+nb+bp}{self}\PYG{o}{.}\PYG{n}{game\PYGZus{}management}\PYG{o}{.}\PYG{n}{servers}
\end{Verbatim}

You then write every thing you want to execute for your game and finish by these lines:

\begin{Verbatim}[commandchars=\\\{\}]
\PYG{n+nb+bp}{self}\PYG{o}{.}\PYG{n}{gui}\PYG{o}{.}\PYG{n}{after}\PYG{p}{(}\PYG{l+m+mi}{100}\PYG{p}{,} \PYG{n+nb+bp}{self}\PYG{o}{.}\PYG{n}{\PYGZus{}update\PYGZus{}map}\PYG{p}{)}
\PYG{n+nb+bp}{self}\PYG{o}{.}\PYG{n}{gui}\PYG{o}{.}\PYG{n}{mainloop}\PYG{p}{(}\PYG{p}{)}
\end{Verbatim}

In the example above, the method \_update\_map will be call every 100ms, and therefore, you will have to override this method (for updating your drawings, check state of the game...).


\chapter{API Documentation}
\label{index:api-documentation}
This part of the documentation details every function, class or method
used for the Robot Table Project.


\section{API Documentation}
\label{api:api-documentation}\label{api::doc}

\subsection{Game}
\label{api:game}\index{Game (class in game)}

\begin{fulllineitems}
\phantomsection\label{api:game.Game}\pysiglinewithargsret{\strong{class }\code{game.}\bfcode{Game}}{\emph{robot}, \emph{sensor}, \emph{network}, \emph{gui}, \emph{game\_management}, \emph{addr\_main\_server}}{}
Create a game.
\begin{quote}\begin{description}
\item[{Parameters}] \leavevmode\begin{itemize}
\item {} 
\textbf{robot} -- Robot object used on the table.

\item {} 
\textbf{sensor} -- Sensor object used to detect IRs.

\item {} 
\textbf{network} -- Network object used to communicate between tables.

\item {} 
\textbf{gui} -- GUI object.

\item {} 
\textbf{game\_management} -- GameManagement object.

\item {} 
\textbf{addr\_main\_server} -- Main server's IP address.

\end{itemize}

\end{description}\end{quote}
\index{draw\_robots() (game.Game method)}

\begin{fulllineitems}
\phantomsection\label{api:game.Game.draw_robots}\pysiglinewithargsret{\bfcode{draw\_robots}}{}{}
Draw robots on the table.

\end{fulllineitems}

\index{get\_addr() (game.Game method)}

\begin{fulllineitems}
\phantomsection\label{api:game.Game.get_addr}\pysiglinewithargsret{\bfcode{get\_addr}}{}{}
Return the ip address of the server.

\textbf{Note}: the solution was find on \href{http://goo.gl/D6FKrq}{Stackoverflow}.

\end{fulllineitems}

\index{is\_main\_server() (game.Game method)}

\begin{fulllineitems}
\phantomsection\label{api:game.Game.is_main_server}\pysiglinewithargsret{\bfcode{is\_main\_server}}{}{}
Return if the server is the main server.

\end{fulllineitems}

\index{load\_map() (game.Game method)}

\begin{fulllineitems}
\phantomsection\label{api:game.Game.load_map}\pysiglinewithargsret{\bfcode{load\_map}}{\emph{path}}{}
Load the map on the canvas.

\end{fulllineitems}

\index{start() (game.Game method)}

\begin{fulllineitems}
\phantomsection\label{api:game.Game.start}\pysiglinewithargsret{\bfcode{start}}{}{}
Launch the game.

\textbf{Note:} This is the method to override when creating your own game.

\end{fulllineitems}

\index{stop() (game.Game method)}

\begin{fulllineitems}
\phantomsection\label{api:game.Game.stop}\pysiglinewithargsret{\bfcode{stop}}{}{}
Stop the game.

\end{fulllineitems}


\end{fulllineitems}

\index{GameManagement (class in game)}

\begin{fulllineitems}
\phantomsection\label{api:game.GameManagement}\pysiglinewithargsret{\strong{class }\code{game.}\bfcode{GameManagement}}{\emph{sensor}, \emph{gui}, \emph{network}, \emph{nb\_servers}}{}
docstring for GameManagement.
\begin{quote}\begin{description}
\item[{Parameters}] \leavevmode\begin{itemize}
\item {} 
\textbf{sensor} -- Sensor object used to detect IRs.

\item {} 
\textbf{gui} -- GUI object.

\item {} 
\textbf{network} -- Network object used to communicate between tables.

\item {} 
\textbf{nb\_servers} -- Number of player/server.

\end{itemize}

\end{description}\end{quote}
\index{do\_calibration() (game.GameManagement method)}

\begin{fulllineitems}
\phantomsection\label{api:game.GameManagement.do_calibration}\pysiglinewithargsret{\bfcode{do\_calibration}}{}{}
Draw 5 crosshairs and then return the calibration's factors
(x\_factors and y\_factors) using the \_calculate\_calibration\_n method.

\end{fulllineitems}

\index{is\_servers\_ready() (game.GameManagement method)}

\begin{fulllineitems}
\phantomsection\label{api:game.GameManagement.is_servers_ready}\pysiglinewithargsret{\bfcode{is\_servers\_ready}}{}{}
Return true if all servers are ready.

\end{fulllineitems}

\index{launch\_game() (game.GameManagement method)}

\begin{fulllineitems}
\phantomsection\label{api:game.GameManagement.launch_game}\pysiglinewithargsret{\bfcode{launch\_game}}{}{}
Launch the game on each server.

\end{fulllineitems}

\index{send\_list\_servers() (game.GameManagement method)}

\begin{fulllineitems}
\phantomsection\label{api:game.GameManagement.send_list_servers}\pysiglinewithargsret{\bfcode{send\_list\_servers}}{}{}
Send a list containing the address of all
servers to each server.

\end{fulllineitems}

\index{start() (game.GameManagement method)}

\begin{fulllineitems}
\phantomsection\label{api:game.GameManagement.start}\pysiglinewithargsret{\bfcode{start}}{\emph{addr\_main\_server}, \emph{is\_main\_server}, \emph{addr}}{}
Execute the needed steps before launching a game.
\begin{quote}\begin{description}
\item[{Parameters}] \leavevmode\begin{itemize}
\item {} 
\textbf{addr\_main\_server} -- Main server's IP address.

\item {} 
\textbf{is\_main\_server} -- True if this is the main server.

\item {} 
\textbf{addr} -- address of the server on which this method is called.

\end{itemize}

\end{description}\end{quote}

\end{fulllineitems}

\index{synchronise\_servers() (game.GameManagement method)}

\begin{fulllineitems}
\phantomsection\label{api:game.GameManagement.synchronise_servers}\pysiglinewithargsret{\bfcode{synchronise\_servers}}{}{}
Synchronise each servers and launch the game.

\end{fulllineitems}

\index{wait\_addr\_servers() (game.GameManagement method)}

\begin{fulllineitems}
\phantomsection\label{api:game.GameManagement.wait_addr_servers}\pysiglinewithargsret{\bfcode{wait\_addr\_servers}}{}{}
Wait that all servers have sent their address.

\end{fulllineitems}

\index{wait\_servers() (game.GameManagement method)}

\begin{fulllineitems}
\phantomsection\label{api:game.GameManagement.wait_servers}\pysiglinewithargsret{\bfcode{wait\_servers}}{}{}
Wait that all servers are ready.

\end{fulllineitems}


\end{fulllineitems}



\subsection{Graphic}
\label{api:graphic}\index{GUI (class in graphic)}

\begin{fulllineitems}
\phantomsection\label{api:graphic.GUI}\pysigline{\strong{class }\code{graphic.}\bfcode{GUI}}
docstring for GUI
\index{after() (graphic.GUI method)}

\begin{fulllineitems}
\phantomsection\label{api:graphic.GUI.after}\pysiglinewithargsret{\bfcode{after}}{\emph{secs}, \emph{function}}{}
Call the function every secs.
\begin{quote}\begin{description}
\item[{Parameters}] \leavevmode\begin{itemize}
\item {} 
\textbf{secs} -- Time interval in seconds.

\item {} 
\textbf{function} -- The function to call.

\end{itemize}

\end{description}\end{quote}

\end{fulllineitems}

\index{coords() (graphic.GUI method)}

\begin{fulllineitems}
\phantomsection\label{api:graphic.GUI.coords}\pysiglinewithargsret{\bfcode{coords}}{\emph{object\_id}, \emph{*coords}}{}
Change and return the coordinates of the object.
\begin{quote}\begin{description}
\item[{Parameters}] \leavevmode\begin{itemize}
\item {} 
\textbf{object\_id} -- Id of the object.

\item {} 
\textbf{*coords} -- (optional) List of coordinate pairs.

\end{itemize}

\end{description}\end{quote}

\textbf{Note:} Same effect as the \href{http://goo.gl/1PMqtC}{Tkinter's method}.

\end{fulllineitems}

\index{delete() (graphic.GUI method)}

\begin{fulllineitems}
\phantomsection\label{api:graphic.GUI.delete}\pysiglinewithargsret{\bfcode{delete}}{\emph{object\_id}}{}
Delete the object with the id object\_id.
\begin{quote}\begin{description}
\item[{Parameters}] \leavevmode
\textbf{object\_id} -- Id's object.

\end{description}\end{quote}

\end{fulllineitems}

\index{init\_graphic() (graphic.GUI method)}

\begin{fulllineitems}
\phantomsection\label{api:graphic.GUI.init_graphic}\pysiglinewithargsret{\bfcode{init\_graphic}}{}{}
Initialize the graphical part using Tkinter.

\end{fulllineitems}

\index{load\_map() (graphic.GUI method)}

\begin{fulllineitems}
\phantomsection\label{api:graphic.GUI.load_map}\pysiglinewithargsret{\bfcode{load\_map}}{\emph{path}}{}
Load the map.
\begin{quote}\begin{description}
\item[{Parameters}] \leavevmode
\textbf{path} -- Path to the map.

\end{description}\end{quote}

\end{fulllineitems}

\index{mainloop() (graphic.GUI method)}

\begin{fulllineitems}
\phantomsection\label{api:graphic.GUI.mainloop}\pysiglinewithargsret{\bfcode{mainloop}}{}{}
Start the graphic mainloop.

\end{fulllineitems}

\index{set\_full\_screen() (graphic.GUI method)}

\begin{fulllineitems}
\phantomsection\label{api:graphic.GUI.set_full_screen}\pysiglinewithargsret{\bfcode{set\_full\_screen}}{}{}
Set the window in full screen.

\end{fulllineitems}


\end{fulllineitems}

\index{Crosshair (class in graphic)}

\begin{fulllineitems}
\phantomsection\label{api:graphic.Crosshair}\pysiglinewithargsret{\strong{class }\code{graphic.}\bfcode{Crosshair}}{\emph{gui}, \emph{x}, \emph{y}, \emph{rad=30}, \emph{color='red'}, \emph{width=2}}{}
Create a Crosshair object.
\begin{quote}\begin{description}
\item[{Parameters}] \leavevmode\begin{itemize}
\item {} 
\textbf{gui} -- GUI object.

\item {} 
\textbf{x} -- x location of the Crosshair.

\item {} 
\textbf{y} -- y location of the Crosshair.

\item {} 
\textbf{rad} -- (optional) Crosshair's radius.

\item {} 
\textbf{color} -- (optional) Crosshair's color.

\item {} 
\textbf{width} -- (optional) Width's lines.

\end{itemize}

\end{description}\end{quote}
\index{delete() (graphic.Crosshair method)}

\begin{fulllineitems}
\phantomsection\label{api:graphic.Crosshair.delete}\pysiglinewithargsret{\bfcode{delete}}{}{}
Delete the crosshair on the canvas.

\end{fulllineitems}

\index{draw() (graphic.Crosshair method)}

\begin{fulllineitems}
\phantomsection\label{api:graphic.Crosshair.draw}\pysiglinewithargsret{\bfcode{draw}}{}{}
Draw the Crosshair on the screen.

\end{fulllineitems}


\end{fulllineitems}

\index{RobotDrawing (class in graphic)}

\begin{fulllineitems}
\phantomsection\label{api:graphic.RobotDrawing}\pysiglinewithargsret{\strong{class }\code{graphic.}\bfcode{RobotDrawing}}{\emph{gui}, \emph{rad=10}, \emph{outline\_color='red'}, \emph{fill\_color='green'}}{}
Create a RobotDrawing object, represented by 3 dots on the screen.
\begin{quote}\begin{description}
\item[{Parameters}] \leavevmode\begin{itemize}
\item {} 
\textbf{gui} -- GUI object.

\item {} 
\textbf{rad} -- (optional) Radius of the dots.

\item {} 
\textbf{outline\_color} -- (optional) Outline color of the dots.

\item {} 
\textbf{fill\_color} -- (optional) Fill color of the dots.

\end{itemize}

\end{description}\end{quote}
\index{draw() (graphic.RobotDrawing method)}

\begin{fulllineitems}
\phantomsection\label{api:graphic.RobotDrawing.draw}\pysiglinewithargsret{\bfcode{draw}}{\emph{leds}}{}
Draw three dots on the screen.

\end{fulllineitems}


\end{fulllineitems}



\subsection{Network}
\label{api:network}\index{Network (class in network)}

\begin{fulllineitems}
\phantomsection\label{api:network.Network}\pysiglinewithargsret{\strong{class }\code{network.}\bfcode{Network}}{\emph{port=5000}}{}
Create a Network object.
\begin{quote}\begin{description}
\item[{Parameters}] \leavevmode
\textbf{port} -- (optional) Port used to communicate.

\end{description}\end{quote}
\index{format() (network.Network method)}

\begin{fulllineitems}
\phantomsection\label{api:network.Network.format}\pysiglinewithargsret{\bfcode{format}}{\emph{addr\_server}}{}
Format the address and return it.

\end{fulllineitems}

\index{get\_irs() (network.Network method)}

\begin{fulllineitems}
\phantomsection\label{api:network.Network.get_irs}\pysiglinewithargsret{\bfcode{get\_irs}}{\emph{addr\_server}}{}
Return the IRs' location of a server.

\end{fulllineitems}

\index{is\_ready() (network.Network method)}

\begin{fulllineitems}
\phantomsection\label{api:network.Network.is_ready}\pysiglinewithargsret{\bfcode{is\_ready}}{\emph{addr\_server}}{}
Return True if the server is ready.

\end{fulllineitems}

\index{launch() (network.Network method)}

\begin{fulllineitems}
\phantomsection\label{api:network.Network.launch}\pysiglinewithargsret{\bfcode{launch}}{\emph{addr\_server}}{}
Launch the game on the server.

\end{fulllineitems}

\index{send\_addr() (network.Network method)}

\begin{fulllineitems}
\phantomsection\label{api:network.Network.send_addr}\pysiglinewithargsret{\bfcode{send\_addr}}{\emph{addr\_server}, \emph{addr\_to\_send}}{}
Send an address to a server and return the response.
\begin{quote}\begin{description}
\item[{Parameters}] \leavevmode\begin{itemize}
\item {} 
\textbf{addr\_server} -- the address to send to.

\item {} 
\textbf{addr\_to\_send} -- the address to send.

\end{itemize}

\end{description}\end{quote}

\end{fulllineitems}

\index{send\_list\_servers() (network.Network method)}

\begin{fulllineitems}
\phantomsection\label{api:network.Network.send_list_servers}\pysiglinewithargsret{\bfcode{send\_list\_servers}}{\emph{addr\_server}, \emph{list\_servers}}{}
Send a list of servers to a server.

\end{fulllineitems}


\end{fulllineitems}



\subsection{Robot}
\label{api:robot}\index{Robot (class in robot)}

\begin{fulllineitems}
\phantomsection\label{api:robot.Robot}\pysiglinewithargsret{\strong{class }\code{robot.}\bfcode{Robot}}{\emph{sensor}, \emph{gui=None}}{}
This class represent a Robot with 3 leds (2 at the back and 1 at the front).
\begin{quote}\begin{description}
\item[{Parameters}] \leavevmode\begin{itemize}
\item {} 
\textbf{sensor} -- Sensor that detect the infrared.

\item {} 
\textbf{robot\_drawing} -- (optional) RobotDrawing object.

\end{itemize}

\end{description}\end{quote}
\index{centre (robot.Robot attribute)}

\begin{fulllineitems}
\phantomsection\label{api:robot.Robot.centre}\pysigline{\bfcode{centre}}
Return a tuple with X and Y location of the robot's centre

\end{fulllineitems}

\index{leds (robot.Robot attribute)}

\begin{fulllineitems}
\phantomsection\label{api:robot.Robot.leds}\pysigline{\bfcode{leds}}
When this property is called, leds position is automatically updated and returned.

Usage:

\begin{Verbatim}[commandchars=\\\{\}]
\PYG{g+gp}{\PYGZgt{}\PYGZgt{}\PYGZgt{} }\PYG{n}{r} \PYG{o}{=} \PYG{n}{Robot}\PYG{p}{(}\PYG{n}{Sensor}\PYG{p}{(}\PYG{p}{)}\PYG{p}{)}
\PYG{g+gp}{\PYGZgt{}\PYGZgt{}\PYGZgt{} }\PYG{n}{r}\PYG{o}{.}\PYG{n}{leds}
\PYG{g+go}{\PYGZob{}\PYGZsq{}front\PYGZsq{}: \PYGZob{}\PYGZsq{}X\PYGZsq{}: 10, \PYGZsq{}Y\PYGZsq{}: 20\PYGZcb{}, \PYGZsq{}left\PYGZsq{}: \PYGZob{}\PYGZsq{}X\PYGZsq{}: 103, \PYGZsq{}Y\PYGZsq{}: 23\PYGZcb{}, \PYGZsq{}right\PYGZsq{}: \PYGZob{}\PYGZsq{}X\PYGZsq{}: 111, \PYGZsq{}Y\PYGZsq{}: 203\PYGZcb{}\PYGZcb{}}
\end{Verbatim}

Note: Since leds position is automatically updated at every call, you should save
leds location in a variable (e.g leds = robot.leds)

\end{fulllineitems}


\end{fulllineitems}



\subsection{Wiimote}
\label{api:wiimote}\index{Wiimote (class in wiimote)}

\begin{fulllineitems}
\phantomsection\label{api:wiimote.Wiimote}\pysiglinewithargsret{\strong{class }\code{wiimote.}\bfcode{Wiimote}}{\emph{width\_resolution=1024}, \emph{height\_resolution=768}, \emph{test=False}}{}
Create a Wiimote object and
connect to the dongle USB.
\begin{quote}\begin{description}
\item[{Parameters}] \leavevmode\begin{itemize}
\item {} 
\textbf{width\_resolution} -- (optional) Width resolution of the sensor

\item {} 
\textbf{height\_resolution} -- (optional) Height resolution of the sensor

\item {} 
\textbf{test} -- (optional) Only use for test-driven

\end{itemize}

\end{description}\end{quote}
\index{connect() (wiimote.Wiimote method)}

\begin{fulllineitems}
\phantomsection\label{api:wiimote.Wiimote.connect}\pysiglinewithargsret{\bfcode{connect}}{}{}
Connect the wiimote to the dongle Bluetooth.

\end{fulllineitems}

\index{disconnect() (wiimote.Wiimote method)}

\begin{fulllineitems}
\phantomsection\label{api:wiimote.Wiimote.disconnect}\pysiglinewithargsret{\bfcode{disconnect}}{}{}
Disconnect the wiimote to the dongle Bluetooth.

\end{fulllineitems}

\index{get\_leds() (wiimote.Wiimote method)}

\begin{fulllineitems}
\phantomsection\label{api:wiimote.Wiimote.get_leds}\pysiglinewithargsret{\bfcode{get\_leds}}{}{}
Return a list of the location of leds detected.

(e.g {[}\{`X': 10, `Y': 20\}, \{`X': 103, `Y': 23\}, \{`X': 111, `Y': 203\}, \{`X': 121, `Y': 13\}{]})

\textbf{Note:} If the location of a led is not detected, X and Y equal -1.

\end{fulllineitems}


\end{fulllineitems}



\chapter{Indices and tables}
\label{index:indices-and-tables}\begin{itemize}
\item {} 
\emph{genindex}

\item {} 
\emph{modindex}

\item {} 
\emph{search}

\end{itemize}



\renewcommand{\indexname}{Index}
\printindex
\end{document}
